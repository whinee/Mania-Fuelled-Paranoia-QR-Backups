\documentclass[../../1_thesis]{subfiles}

\begin{document}

\chapter{The Problem and Its Background}

\section{Introduction}

This study explores the design and implementation of a human-decodable, QR-based archival preservation system. As digital storage media face inevitable decay and format obsolescence, the need for an archival method that bridges human readability and machine precision becomes increasingly critical. The proposed system encodes data into QR codes arranged for redundancy, legibility, and long-term recoverability, ensuring that essential knowledge remains accessible even in the absence of functioning computers.

\section{Background of the Study}

Modern data storage relies heavily on digital formats---such as DOCX, XLSX, PDFs---and digital medium of storage such as hard drives and cloud storage. While convenient, these are fragile in the face of time: bit rot, format incompatibility, and dependency on specific hardware and software ecosystems threaten their longevity. Historically, human civilizations relied on physical inscriptions—stone tablets, manuscripts, and printed texts—that survived millennia. In contrast, digital information may not last beyond a few decades without continuous migration. This study seeks to integrate the endurance of physical archives with the density and structure of modern digital encoding.

\section{Theoretical Framework}

The theoretical basis of this research lies in three main domains:
\begin{enumerate}
\item \textbf{Information Theory} — for understanding data encoding efficiency and error correction.
\item \textbf{Archival Science} — for studying the principles of long-term preservation and access.
\item \textbf{Semiotics and Human Factors} — for ensuring that the encoded information remains interpretable by humans without computational assistance.
\end{enumerate}
By merging these disciplines, the research aims to create an archival format that is both symbolic and functional—a literal preservation of meaning across generations.

\section{Conceptual Framework}

The conceptual framework envisions a dual-readable archival system where information is stored in a two-tier format:

\begin{enumerate}
\item \textbf{Machine Layer:} QR codes optimized for digital scanning and automated decoding.
\item \textbf{Human Layer:} A set of visual and linguistic patterns that allow a trained reader to manually interpret and reconstruct data using pen, paper, and logic alone.
\end{enumerate}

This layered approach aligns with the idea of technological fallback—ensuring information remains recoverable even under total computational collapse.

\section{Statement of the Problem}

Current archival systems prioritize convenience and machine efficiency at the expense of long-term readability. The central problem addressed by this study is the lack of a unified, low-cost archival medium that:
\begin{enumerate}
\item Can be both machine-readable and human-decodable.
\item Uses open and accessible encoding standards.
\item Remains recoverable without dependence on specific software or hardware.
\end{enumerate}

\section{Objectives of the Study}

The study aims to:
\begin{enumerate}
\item Design a QR-based archival encoding scheme suitable for long-term preservation.
\item Evaluate the human-decoding feasibility of the proposed format.
\item Develop Python-based tools for generating and decoding such archives.
\item Demonstrate a full archival workflow, from encoding a document to recovering it from printed QR codes.
\end{enumerate}

\section{Significance of the Study}

This study contributes to both digital preservation and alternative communication methods. It offers:

\begin{enumerate}
\item A resilient archival solution that merges analog and digital traditions.
\item Educational value in demonstrating the intersection of computation, linguistics, and manual reasoning.
\item A foundation for future research on human-readable encodings, post-digital preservation, and low-tech data recovery.
\end{enumerate}

\section{Scope and Limitations of the Study}
The system focuses on the encoding and decoding of small- to medium-sized datasets (e.g., text documents, structured tables) using QR code-based grids. Image and multimedia encoding are outside the scope of this work. While human readability is prioritized, it assumes literacy in basic symbolic logic and access to decoding instructions. Durability testing of physical media is limited to conceptual discussion rather than empirical environmental studies.

\section{Definition of Terms}
\begin{description}
\item[Archival Preservation] The process of maintaining and protecting information for long-term accessibility.
\item[QR Code] A two-dimensional barcode capable of storing alphanumeric and binary data with error correction.
\item[Human-Decodable] Data that can be interpreted manually without computational tools.
\item[Error Correction] A method of recovering lost or damaged data within an encoded message.
\item[Digital Obsolescence] The phenomenon where digital formats or storage media become unusable due to technological advancement or decay.
\end{description}

\end{document}
