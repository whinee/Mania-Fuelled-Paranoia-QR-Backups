\documentclass[../1_thesis]{subfiles}

\begin{document}

\chapter*{Preface}
\addcontentsline{toc}{chapter}{Preface}

This research began not as a formal study, but as a coping mechanism. I live with Bipolar I disorder, a condition that can be both ecstatic and devastating, in what I call \textit{"Ecstatic Despair"}. It brings moments of boundless energy and creativity, followed by periods of deep-seated melancholy. With it come bouts of paranoia about loss---loss of memory, loss of meaning, and at times, visions of a total technological collapse where all digital traces vanish. I have lost countless files over the years through carelessness and accident, and each loss felt like a small death: moments, creations, and fragments of self, gone forever. Out of that grief came the impulse to preserve---to find a way to make memory tangible again.

That impulse led me to discover QR-based backup systems, which fascinated me but left me unsatisfied. Most assumed that the world's infrastructure would endure---that servers, scanners, and networks would always be there to read the data. But what if they were not? What if preservation itself had to survive the end of everything familiar? This question became the seed of what would later grow into \textit{Remember, Record, Resist: A Self-Contained QR-Code Archival System for Long-Term Information Preservation}.

Two months into exploring this idea, I realized it wasn't just a side project anymore. It was a statement---a resistance to loss, not only personal but collective. In a country where archives are burned, histories are rewritten, and memories of war, dictatorship, and abuse are brushed aside, the act of preserving data becomes an act of defiance.

This study is for those who remember when they are told to forget; for those who lost their voices to erasure---whether through the silence of the dead, the neglect of the living, or the convenience of indifference. This research is dedicated to the archivists, historians, churches, and survivors who keep fragments of truth alive in fragile paper and fading ink. To them, and to anyone who has ever feared that everything might one day be lost---this is my small resistance.

\textit{Mabuhay ang katotohanan! Mabuhay ang mga alaala ng kahapon!}

\textit{Mabuhay ang paghihimagsik ng alaala laban sa pagkalimot!}

\end{document}